\documentclass[reqno]{amsart}
\usepackage{mathtools,upref,siunitx,upquote,fancyvrb,color}
\usepackage{cleveref}
\usepackage[hyphens]{url}
\usepackage[utf8]{inputenc}
\input{FJHDef}

\newtheorem{theorem}{Theorem}

\usepackage{algpseudocode}
\usepackage{algorithm, algorithmicx}
\algnewcommand\algorithmicparam{\textbf{Parameters:}}
\algnewcommand\PARAM{\item[\algorithmicparam]}
\algnewcommand\algorithmicinput{\textbf{Input:}}
\algnewcommand\INPUT{\item[\algorithmicinput]}
\algnewcommand\RETURN{\State \textbf{Return }}

\DeclareMathOperator{\distdens}{\overset{\text{PDF}}{\sim}}

\textwidth 6.5in
\hoffset -0.7in


\allowdisplaybreaks %allow equations to span multiple pages


\begin{document}
\title{The Duality Between Functions and Measures in Reproducing Kernel Hilbert Spaces}
\author{Fred J. Hickernell}


\maketitle

%%%%%%%%%%%%%%%%%%%%%%%%%%%%%%%%%%%%%%%%%%%%%%%%%%%%%%%
\section{Duality in $\reals^d$}
%%%%%%%%%%%%%%%%%%%%%%%%%%%%%%%%%%%%%%%%%%%%%%%%%%%%%%%

%%%%%%%%%%%%%%%%%%%%%%%%%%%%%%%%%%%%%%%%%%%%%%%%%%%%%%%
\subsection{Vectors and Matrices}
%%%%%%%%%%%%%%%%%%%%%%%%%%%%%%%%%%%%%%%%%%%%%%%%%%%%%%%

Let $(\reals^d,\ip[\cf]{\cdot}{\cdot})$ be the Hilbert space defined by the inner product 
\begin{equation*}
    \ip[\cf]{\vf}{\vg} := \vf^T \mA \vg \qquad \forall \vf, \vg \in \reals^d.
\end{equation*}
where $\mA$ is a symmetric, strictly positive definite matrix.  Let $\mK :=  \mA^{-1}$.  We also define a Hilbert space $(\reals^d,\ip[\cm]{\cdot}{\cdot})$ by the inner product
\begin{equation*}
    \ip[\cm]{\vmu}{\vnu} := \vmu^T \mK \vnu \qquad \forall \vmu, \vnu \in \reals^d.
\end{equation*}
Note that there is a \emph{one-to-one and onto, isometric(?) mapping} $\vT: \reals^d \to \reals^d$ defined by 
\[
\vT(\vmu) := \mK \vmu
\]
such that 
\begin{equation*}
    \ip[\cf]{\vT(\mu)}{\vT(\vnu)} = \bigl(\vT(\mu)\bigr)^T \mA \bigl(\vT(\nu)\bigr) = \vmu^T \mK \mA \mK \vnu = \vmu^T \mK \vnu = \ip[\cm]{\vmu}{\vnu} \qquad \forall \vmu, \vnu \in \reals^d.
\end{equation*}


If $\vK_1, \ldots, \vK_d$ denote the columns of $\mK$, then there is a \emph{reproducing property}:
\begin{equation*}
    \ip[\cf]{\vK_x}{\vf} = \vK_x \mA \vf = \ve_x^T \vf = f_x \qquad \forall \vf \in \reals^d, \ x \in \{1, \ldots, d\},
\end{equation*}
where $\ve_x := (0, \ldots, 0,\underbrace{1}_{x^{\text{th}} \text{ position}}, 0 \ldots, 0)^T$ and $\vf = (f_1, \ldots, f_d)^T$.  In other words, $ \ip[\cf]{\vK_x}{\vf}$ reproduces the $x^{\text{th}}$ element of $f$.

%%%%%%%%%%%%%%%%%%%%%%%%%%%%%%%%%%%%%%%%%%%%%%%%%%%%%%%
\subsection{Functions and Measures}
%%%%%%%%%%%%%%%%%%%%%%%%%%%%%%%%%%%%%%%%%%%%%%%%%%%%%%%
Let's now rewrite these ideas in terms of functions and measures.  Let $\Omega = \{1, \ldots, d\}$, and let $\ce = 2^\Omega$.  Define a vector space of functions with it's inner produce:
\begin{equation*}
    \cf : = \{f : \Omega \to \reals\}, \qquad \ip[\cf]{f}{g} := \sum_{t,x=1}^d f(t) A(t,x) g(x),
\end{equation*}
where $\bigl( A(t,x) \bigr)_{t,x = 1}^d$ is a symmetric, positive definite matrix with inverse $\bigl( K(t,x) \bigr)_{t,x = 1}^d$.  This means that
\begin{equation} \label{eq:AKeqI}
    \sum_{t=1}^d A(s,t)K(t,x) = \begin{cases} 1, & s=x, \\ 0, & s\ne x. \end{cases}
\end{equation}

Define a vector space of measures with it's inner product:
\begin{equation*}
    \cm : = \{\text{all measures } (\Omega, \ce, \mu) \}, \quad 
    \ip[\cm]{\mu}{\nu} := \int_{\Omega \times \Omega} K(t,x) \, (\mu \times \nu) (\dif t \times \dif x) = \sum_{t,x=1}^d \mu(\{t\}) K(t,x) g(\{x\}),
\end{equation*}

Note that 
\[
K(\cdot,t) \in \cf \qquad \forall t \in \Omega.
\]
Moreover, we have the reproducing property:
\begin{align*}
\ip[\cf]{K(\cdot,s)}{f} &=  \sum_{t,x=1}^d K(t,s) A(t,x) f(x) 
= \sum_{x=1}^d f(x) \biggl[\sum_{t}^d K(t,s) A(t,x) \biggr] \\
& = f(s) \qquad \text{by \eqref{eq:AKeqI}}.
\end{align*}


Define the \emph{one-to-one and onto, isometric(?) mapping} $T : \cm \to \cf$ by 
\begin{gather*}
    T(\mu)(x) := \ip[\cf \times \cm]{K(\cdot, x)}{\mu} = \int_\Omega K(t, x) \, \mu(\dif t) = \sum_{t=1}^d K(t,x) \mu(\{t\}) \qquad \forall x \in \Omega, \mu \in \cm, \\
    \intertext{where $ \ip[\cf \times \cm]{\cdot}{\cdot} : \cf \times \cm \to \reals$ is a real, bilinear function defined by} 
    \ip[\cf \times \cm]{f}{\mu} := \int_\Omega f \, \mu(\dif x) = \sum_{x=1}^d f(x) \mu(\{x\}) \qquad \forall f \in \cf, \ \  \mu \in \cm.
\end{gather*}
The following duality relationship holds:  for all $\mu, \nu \in \cm$, if $f = T(\mu)$ and $g = T (\nu)$, then 
\begin{align*}
    \ip[\cf]{f}{g} 
    & = \sum_{t,x=1}^d f(t) A(t,x) g(x)
    = \sum_{s,t,x=1}^d f(t) A(t,x)  K(s,x) \nu(\{s\})  \\
    &
    = \sum_{s,x=1}^d \biggl [ f(t) \nu(\{s\}) \sum_{t=1}^d A(t,x)  K(s,x) \biggr]
    = \sum_{t=1}^d f(t) \nu(\{t\}) = \ip[\cf \times \cm]{f}{\nu} \\
    & = \sum_{s,t=1}^d K(s,t) \mu(\{s\}) \nu(\{t\}) =  \ip[\cm]{\mu}{\nu} .
\end{align*}

We want to generalize these ideas to 
\begin{itemize}
    \item general Hilbert spaces, $(\cf, \ip[\cf]{\cdot}{\cdot})$  of functions $f : \Omega \to \reals$ with reproducing kernels, $K$, and 
    \item corresponding Hilbert spaces, $(\cm, \ip[\cm]{\cdot}{\cdot})$  of measures $(\Omega, \ce, \mu)$.
\end{itemize}
The interesting cases will be where $\cf$ is infinite dimensional.



%%%%%%%%%%%%%%%%%%%%%%%%%%%%%%%%%%%%%%%%%%%%%%%%%%%%%%%
\section{Duality in Arbitrary Reproducing Kernel Hilbert Spaces}
%%%%%%%%%%%%%%%%%%%%%%%%%%%%%%%%%%%%%%%%%%%%%%%%%%%%%%%

\begin{theorem} Suppose that $K: \Omega \times \Omega \to \reals$ a strictly positive definite function.  Then the following are true:
\begin{enumerate}
\renewcommand{\labelenumi}{\roman{enumi})}

    \item There exists a unique reproducing kernel Hilbert space, $\bigl(\cf,\ip[\cf]{\cdot}{\cdot}\bigr)$ of functions defined on $\Omega$ with reproducing kernel $K$.
    
    \item There exists a unique Hilbert space, $\bigl(\cm,\ip[\cm]{\cdot}{\cdot}\bigr)$, of measures defined on $\Omega$ with $\sigma$-algebra $\ce$, where $\ce$ contains all sets consisting of a single element of $\Omega$, and with
    \begin{equation} \label{eq:Mipdef}
    \ip[\cm]{\mu}{\nu} := \int_{\Omega \times \Omega} K(t,x) \, (\mu \times \nu) (\dif t \times \dif x).
    \end{equation}
    Uniqueness means that if $\bigl(\cm',\ip[\cm]{\cdot}{\cdot}\bigr)$ is Hilbert space of measures with the identical inner product, then $\cm \setminus \cm' = \emptyset$ and for all $\lambda \in \cm' \setminus \cm$, there exists a $\mu \in \cm$ such that $\norm[\cm]{\lambda - \mu} = 0$.
    
    
    \item \label{thmiii} There exists a \emph{one-to-one and onto, isometric(?) mapping} $T : \cm \to \cf$ defined as 
    \begin{subequations}
\begin{gather}
    T(\mu)(x) := \ip[\cf \times \cm]{K(\cdot, x)}{\mu} = \int_\Omega K(t, x) \, \mu(\dif t) \qquad \forall x \in \Omega, \mu \in \cm, \\
    \intertext{where $ \ip[\cf \times \cm]{\cdot}{\cdot} : \cf \times \cm \to \reals$ is a real, bilinear function defined by} 
    \ip[\cf \times \cm]{f}{\mu} := \int_\Omega f(x) \, \mu(\dif x) \qquad \forall f \in \cf, \ \  \mu \in \cm.
\end{gather}
\end{subequations}



\item For any $\mu, \nu \in \cm$,
\begin{equation}
    \ip[\cm]{\mu}{\nu} = \ip[\cf \times \cm]{T(\mu)}{\nu} = \int_\Omega T(\mu)(x) \, \nu(\dif x) = \ip[\cf]{T(\mu)}{T(\nu)},
\end{equation}
which 
\end{enumerate}

\end{theorem}

Interpretation of this theorem.
\begin{itemize}
    \item iii) means that the measures act as the coefficients determining the functions using the basis given by the reproducing kernel.
    \item iv) implies that the inner product of two functions is equal to the appropriate inner produce of their coefficients.  
    \item Since the Riesz Representation Theorem implies that all linear functionals on Hilbert spaces may be represented as inner products of with a representer, iv) imples that they can also be represented as integrals with respect to a particular measure.
\end{itemize}

\begin{proof}
Needed.


i) is known

ii) is kind of in \cite{Hic99a}

iii) and iv) I have not seen
\end{proof}


%%%%%%%%%%%%%%%%%%%%%%%%%%%%%%%%%%%%%%%%%%%%%%%%%%%%%%%
\section{Open Questions}
What does this give us?

I may have some ideas
%%%%%%%%%%%%%%%%%%%%%%%%%%%%%%%%%%%%%%%%%%%%%%%%%%%%%%%




\bibliographystyle{amsplain}
\bibliography{FJH23,FJHown23}

\end{document}
